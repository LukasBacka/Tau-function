\section{Zobecnění tau funkce}

Vytvoříme matematický aparát pro vyjádření kořenů
rovnice \(T(\xi_k | x) =x\). Tuto funkci pojmenujeme
Super-tau funkcí, jejíž notace činí \(\mathbb{T}(\xi_k | x)\).

\begin{definition}[Super-Tau funkce]
      Super-Tau funkce je zobecnění \(\tau\)-funkce s \(\xi\) 
      vektorem. Můžeme ji charakterizovat následující identitou
      \[\mathbb{T}(\xi_k | \mathcal{T}(\xi_k | x) - x ) = x\]
      kde \(\xi_k = (a_1, a_2, ..., a_n), a_1, a_2, ..., a_n \in \mathbb{C}\).
\end{definition}

Pro Super-tau funkci platí následující vztahy

\begin{corollary}
      Jestliže \(\mathbb{T}(\xi_k | x)\) je super-tau 
      funkce, potom dle výše uvedené definice platí
      \begin{align}
            \mathbb{T}(\xi_k |0) &=\tau(\xi_k) \\
            x\mathbb{T}(e^{-x} | 0) &= W(x) \\
            \mathbb{T}(n, m) &= - \frac{W(-n^{-m}
            \ln(n))}{\ln(n)} - m
      \end{align}
\end{corollary}

Prostřednictvím Langrangeovy expanze 
odvodíme následující expanzi Super-Tau funkce

\begin{align}
      \mathbb{T}(\xi_k | x) = \sum_{n\geq 1}(x -
      \mathcal{T}(\xi_k | 0))^n \lim_{\omega \to 
      0}  \frac{\omega^{n+1}}{(\mathcal{T}(\xi_k 
      | \omega) - \mathcal{T}(\xi_k | 0) - \omega)^n}
\end{align}
