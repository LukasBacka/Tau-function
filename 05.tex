\section{Otázky}

\subsection{Tau funkce}

\begin{conjecture}[Gaussova racionální řádová tau funkce]
      Podmnožinou oboru hodnot řadové tau funkce \(H_\tau\) 
      jsou prvky \(\phi\), které splňují podmínky
      \begin{enumerate}
            \item \((\forall \phi)(\phi = \tau(x))(x \neq -1)(Im(\phi) \neq 0)\)
            \item \((\forall \phi)(\phi \notin \mathbb{Q}[i])\)
      \end{enumerate}
\end{conjecture}

Tuto hypotézu můžeme zobecnit na více systémů.

\begin{conjecture}
      Řadová tau funkce \(\tau[x]\) řádu \(x, x \in 
      \mathbb{Z}^+\), má vždy \(x - 1\) různých hodnot
      v principiální části komplexní roviny.
\end{conjecture}

\begin{conjecture}
      Existuje nekonečně mnoho prvků oboru hodnot \(H_ \tau\), 
      které jsou přidruženy \(\tau[x]\) v bodě \(x\), a jsou si 
      vzájemně sdružené, \(\tau[x] = \phi, \tau[x] = \phi^*.\)
\end{conjecture}

\subsection{G-funkce}

\begin{conjecture}
      Existuje nekonečně mnoho nulových kapslý pro 
      \(U(s)\), kde \(s \geq 0\), kdy \(G(U(s-1)) <
      G(U(s))\); \(G(U(s+1)) < G(U(s))\).
\end{conjecture}
 
\begin{conjecture}
      Nulová kapsle pro \(U(2)\) je jediná splňující 
      vztah \(G(U(s-1)) < G(U(s))\); \(G(U(s+1)) 
      < G(U(s))\), kde s \(\geq 0\).
\end{conjecture}
 
 \begin{conjecture}
       Neexistují nulové kapsle pro \(U(s)\), 
       \(s < 0,\) pro které platí \(|s|>G(U(s))\).
 \end{conjecture}
 
 \begin{conjecture}
       Platí vztahy
       \begin{align}
             (U(s) > G)(U(s)) &, s < 0 \\
             ( U(s) < G)(U(s)) &, s \geq 0
       \end{align}
 \end{conjecture}
 
 \subsection{Skokové body}
 
\begin{conjecture}
      Neexistuje nekonečně mnoho skokových bodů.
\end{conjecture}

\begin{question}
      Nechť máme skokový pár, který je tvořen dvojicí nejbližšího 
      skokového bodu lokálního minima \((J(min))\) a skokového bodu 
      lokálního maxima \((J(Max) )\), potom nechť \(\chi\) je definována
      \[\chi = \sqrt{|J(min) - U(J(min))|^2 + |J(Max) - U(J(Max))|^2}\]
      Existuje minimální a maximální hodnota \(\chi\)?
\end{question}
