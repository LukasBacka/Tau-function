\section{U funkce}

U funkce je analytická kontinuální funkce 

\begin{definition}[U funkce]
      U funkcí rozumíme vztah
      
      \begin{align}
            U(j) = \sum_{n=2}^{\infty} \frac{1}{\eta_n(j)}
      \end{align}
      
      kdy \(U(j)\) je U funkce s j. řádem.
\end{definition}

Tuto funkci můžeme aproximovat vztahem \(U(j) 
\cong 10^{-0.28074j + 0.1229} := \tilde{U}(j)\) 
pro \(j \in \mathbb{R}\). Derivace U-funkce činí

\begin{align}
      U'(j) = - \sum_{n=2}^\infty \frac{1}{\eta^2_n(j)}
      \prod_{r=2}^n \left[\eta_{(r; n)}(j) \ln(r) \right]
\end{align}

\begin{table}[h!]
      \centering
      \begin{tabular}{||c c c c||} 
            \hline
            k & U(k) & \(\tilde{U}(k)\) & G(k) \\ [0.5ex] 
            \hline\hline
            -5 & 33,6216... & 33.6202...& 1  \\
            -4 & 17,61436...& 17.6141... & 1 \\
            -3 & 9,59052... & 9.22826... & 1 \\
            -2 & 5,511116... & 4.83482... & 1 \\
            -1 & 3,24746 & 2.53303... & 1 \\
            0  & 1,625...  & 1.327089... & 21  \\
            1 & 0,625... & 0.695280... & 21 \\
            2  & 0,251953...  & 0.364267... & \(12 \ 958 \ 345\)  \\
            3 & 0,125... & 0.190845... & 5 \\
            4  & 0,0625... & 0.0999862...  & 20  \\
            5  & 0,03125... & 0.0523842... & 68  \\
            6  & 0, 015625... & 0.0274448...  & 213  \\ [1ex] 
            \hline
      \end{tabular}
\end{table}

Vložíme-li hodnoty U-funkce a 
aproximace U-funkce do tabulky získáme

\newpage

\begin{definition}[G-funkce]
      \(G\) funkce je počet nul v první řadě nul hodnoty
      j-té U funkce. například \(G(0.25000...) = \infty,\)
      \(G(1.0000012) = 5\), \(G(0.00020001) = 3\)
\end{definition}

\subsection{Error v aproximaci U-funkce}

Pro určení chybovosti funkce \(\tilde{U}(j)\), která 
je aproximací pro U-funkci zvolíme několik bodů a určíme
absolutní hodnotu rozdílu hodnot obou funkcí, přičemž budeme
rozumět notaci \(\varepsilon(j)\) chybovost v bodě j a 
\(\varepsilon(j\%)\) chybovost v procentech. Potom absolutní
error určíme vztahem

\begin{align}
      \varepsilon(j) = | U(j) - \tilde{U}(j) 
      |; \  & \  U(j) \neq 0 
\end{align}

Relativní error můžeme určit

\[\eta = \frac{\varepsilon(j) }{|U(j)|} = \frac{ | U(j) -
\tilde{U}(j) |}{|U(j)|} = \left| 1 - 
\frac{\tilde{U}(j)}{U(j)}\right|\]
tedy

\[\varepsilon(j\%) = 100 \% \times 
\eta = 100\% \left| 1 - \frac{\tilde{U}(j)}{U(j)}\right|\]
Vložíme-li hodnoty do tabulky

\begin{table}[h!]
      \centering
       \begin{tabular}{||c c c c c c||} 
             \hline
             k & U(k) & \(\tilde{U}(k)\) &  \(\varepsilon(j)\)  & \(\eta\) & \(\varepsilon(j\%)\)  \\ [0.5ex] 
            \hline\hline
            -5 & 33,6216... & 33.6202... & 0.0014 & 0.00004164 & 0.00416399 \% \\
            -4 & 17,61436...& 17.6141... & 0.00026 & 0.00001476 & 0.001476 \%\\
                  -3 & 9,59052... & 9.22826...& 0.36226 & 0.03777272 & 3.777272 \% \\
            -2 & 5,511116... & 4.83482...&  0.676296 & 0.12271489 & 12.271489 \% \\
            -1 & 3,24746 & 2.53303... & 0.71443 & 0.21999655 & 21.999655 \% \\
            0  & 1,625...  & 1.327089..& 0.297911 & 0.18332985 & 18.332985 \% \\
                  1 & 0,625... & 0.695280...& 0.07028 & 0.112448 & 11.2448 \% \\
            2  & 0,251953...  & 0.364267... & 0.112314 & 0.44577362 & 44.577362 \% \\
            3 & 0,125... & 0.190845...  & 0.065845 & 0.52676 & 52.676 \% \\
            4  & 0,0625... & 0.0999862...  & 0.0374862 & 0.5997782 & 59.97782 \%\\
            5  & 0,03125... & 0.0523842... & 0.0211342 & 0.6762944 & 67.62944 \% \\
            6  & 0,015625... & 0.0274448...  & 0.0118198 & 0.7564672 & 75.64672 \% \\ [1ex] 
            \hline
      \end{tabular}
\end{table}

z tabulky můžeme určit, že přesnost aproximace U funkce
v bodě k se snižuje se zvyšující hodnotou k.

\subsection{Obsah plochy U funkce na kladné straně}

Obsah U funkce na kladné straně je jistě konečný. Tuto plochu
budeme nazývat plocha G \(:=\mathbb{G}\). Plocha \(\mathbb{G}\) 
může být aproximována aproximací \(\tilde{U}(j)\),

\begin{align}
      \mathbb{G} &\cong \int_0^\infty \tilde{U}(j) dj =\\
      &= \int_0^\infty  10^{-0.28074j + 0.1229} dj =\\
      &= \int_0^\infty e^{-\frac{14037 \ln(10)}{50 \ 000}s -
      \frac{1229 \ln(10)}{10 \ 000}} ds= \\
      &= \frac{10^{4.1229}\times 5}{14037 \ln(10)} =\\
      &\cong 2.05296
\end{align}

potom

\begin{align}
      \mathbb{G} &= \int_0^{\infty} U(s) ds =\int_0^\infty 
      \sum_{n=2}^{\infty} \frac{1}{\eta_n(s)} ds =\\
            &=\sum_{n=2}^{\infty} \int_0^\infty \frac{1}{\eta_n(s)}
      ds = \sum_{n=2}^\infty [J_n(s) ]_{s=0}^{s=\infty} =\\
      &= \sum_{n=2}^\infty \lim_{s \to \infty} J_n(s) - J_n(0) 
      = -\sum_{n=2}^\infty J_n(0) 
\end{align}

kde

\[J_n(s) = \int \frac{1}{\eta_n(s)} ds\]

například 

\begin{align}
      J_2(s) &= -\frac{2^{-s}}{\ln(2)} + 
      c & J_2(0) = -\frac{1}{ln(2)} = -1.4427... \\
      J_3(s) &= \frac{Ei(-3^s \ln(2))}{\ln(3)} 
      + c & J_3(0)=  \frac{Ei(-ln(2))}{ln(3)} = -0.344681 \\
      J_4(s) &= \mbox{??} & J_4(0) = \mbox{??} \\
      \lim_{\xi \to \infty} J_{k}(\xi) &= 0
\end{align}

Bohužel \(J_4(x)\) nemůže být zapsána prostřednictví 
elementárních funkcí, takže nemůže být lehce zjištěna
hodnota pro \(J_x(0)\), pro \(x \geq 4\).
      
\begin{question}
      Jaká je hodnota následujícího výrazu?
      \[-\sum_{n=2}^\infty J_n(0) \]
\end{question}

\subsection{U funkce v \(\mathbb{C}\)}

\(U(j)\) pro \(j \in \mathbb{R} \) má vlastnosti 
exponenciální funkce, která může být aproximována 
exponenciální funkcí \(10^{-0.28074j + 0.1229}\), 
bohužel pro body \(s \in \mathbb{C} \setminus 
\mathbb{R}\) v \(U(s)\) je tato aproximace 
nevyužitelná. U-funkce splňuje následující vztah
\(U(s) = \overline{U(\overline{s})}\). Toto implikuje,
že je funkce symetrická podle reálné osy, jinými slovy
U-funkce je sudá. Potom kladnou principiální částí
U-funkce myslíme kladnou část reálné roviny oboru 
hodnot funkce U a zápornou principiální částí část
záporné reálné osy. Vytvořili by jsme-li graf 
imaginárních částí prvků \(p, p \in H_U\) všimli
by jsme si tzv. skokových bodů.

\begin{definition}[Skokové body]
      Skokovými bod rozumíme body \(J = U(j),\) pro které platí
      
      \begin{align}
            \frac{d}{dj}\left(\Re(U(j))\right) =0  
      \label{M1}
      \end{align}

      nbo
      
      \begin{align}
             \frac{d}{dj}\left(\Im(U(j))\right) = 0
      \label{M2}
      \end{align}
      
      Skokové body splňující (\ref{M1}) nazýváme reálné 
      skokové body a (\ref{M2}) Imaginární skokové body.
      
\end{definition}

Tyto body se vyznačují extrémní změnou průběhu funkce. 
Zatím byl nalezen pouze jeden reálný a imaginární skokový
bod na kladné principiální části U-funkce, toto implikuje 
existenci těchto bodů i na záporné principiální části 
U-funkce. Imaginární skokové body značíme \(J(Im)\) a 
reálné skokové body \(J(Re)\), potom \(U(J(Re))\) a 
\(U(J(Im))\) jsou hodnoty těchto bodů.

\begin{table}[h!]
      \centering
      \begin{tabular}{||c c||} 
            \hline
            J(Re) & U(J(Re)) \\ [0.5ex] 
            \hline\hline
            0.0001+1.8i & \(-5.14473 \times 10^{39} -9.83268 \times 10^{39} i\)  \\
            0.0001 - 1.8i & \(-5.14473 \times 10^{39} +9.83268 \times 10^{39} i\) \\ [1ex] 
            \hline
      \end{tabular}
\end{table}

\begin{table}[h!]
      \centering
      \begin{tabular}{||c c||} 
            \hline
            J(Im) & U(J(Im)) \\ [0.5ex] 
            \hline\hline
            1.8i & \(-4.64309 \times 10^{39} -1.61765  \times 10^{40} i\)  \\
            -1.8i & \(-4.64309 \times 10^{39} +1.61765  \times 10^{40} i\)  \\ [1ex] 
            \hline
      \end{tabular}
\end{table}
