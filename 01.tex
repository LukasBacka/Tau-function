\section{Úvod}

Tetrace, nebo iterativní tetrace je čtvrtá hyperoperace
\cite{24} po sčítání, součinu a exponenciální funkci. 
Tetrace je definována jako 

\[^nr := \underbrace{r^{r^{\iddots^{r}}}}_{\text{n}} \]

respektive kopie r jsou exponentovány n-krát, 
rekurzivně ji definujeme

\begin{align}
      ^n r :=\left\{ 
      \begin{array}{ll}
            1 & (n = 0) \\
            r^{\left( ^{n-1}r
            \right)} & (n \geq 1)
      \end{array} \right.
\end{align}

zřejmě \(^{n-1}r = \log_r(^n r)\). Na parametry r a
a je referováno jako na základ a výšku. V tomto dokumentu
využijeme speciální notaci pro vyjádření tetrace

\begin{align}
      ^n r:= \mathcal{T}(\underbrace{r, r, ..., 
      r}_{\text{n}} | 1) := \exp_r^n 1
\end{align}

pro jednoduchost budeme zapisovat vektor ve tvaru 
\((r_1, r_2, ..., r_x)\) jako \(\xi_x\). Můžeme tento
vektor považovat za uspořádanou x-tici, pro kterou
vytvoříme obecnější notaci \cite{28}

\[\xi_k: = \bigotimes_{i=1}^k a_i \]

poté definujeme odčítání prvků v \(\xi\) vektoru
\begin{definition}[Odčítání \(\xi\) vektorů]
      Nechť máme \(\xi_k\), kde \(\xi_k :=\bigotimes_{i=1}^k 
      a_i := (a_1, a_2, ..., a_k)\), pro \(a_1, a_2, ... \in 
      \mathbb{C}\). Odčítání prvku \(a_n\) od \(\xi\) vektoru,
      kde \(k \geq n \geq 1\) definujeme:

      \begin{align}
            \xi_k \setminus a_n &:= \bigotimes_{\{k
            \geq j \geq 1\} \setminus n} a_i
      \end{align}
\end{definition}

Budeme využívat speciální notace pro nejvýznamnější prvky:
\(a_1 = \alpha\), \(a_2 = \beta\), \(a_3 = \gamma\), \(a_4 
= \delta\). Poslední prvek uspořádaného ksi vektoru budeme 
značit \(\omega\). Nicméně \(\mathcal{T}\)-funkci zobecníme
a definujeme následujícím způsobem:

\begin{definition}
      \(\mathcal{T}\) funkce je rekurzivní funkce \(\mathcal{T}(n_1,
      n_2, ..., n_k | x) = n_1^{\mathcal{T}(n_2, ..., n_k | x)}\) s
      konečným bodem \(\mathcal{T}(n_k | x ) = n_k^x\) , kde koeficienty
      \((n_1, n_2, ..., n_k)\) tvoří \(\xi\) vektor, kdy platí \(n_1, 
      n_2, ..., n_k \in  \mathbb{C}\).  \(\mathcal{T}(n_1, n_2, ..., 
      n_k | x) \) je \(\mathcal{T}\) funkce k-tého řádu s bazí \(x\).
\end{definition}

Potom je nutné definovat její inverzní funkci, jelikož rovnici 
\(\mathcal{T}(2, 3 | x) = x\)  můžeme přeformu-lovat do 
ekvivalentního tvaru \(log_3(log_2(x)) = x\).

\begin{definition}[\(\mathcal{D}\)-funkce]
      \(\mathcal{T}^{-1}\) funkce je rekurzivní funkce 
      
      \begin{equation}
            \mathcal{T}^{-1}(n_1, n_2, ..., n_k | x) 
            = log_{n_1}({\mathcal{T}^{-1}(n_2,
            ..., n_k | x)})
      \label{3}
      \end{equation}
      
      s konečným bodem \(\mathcal{T}^{-1}(n_k | x ) 
      = log_{n_k}(x)\) , kde \(n_1, n_2, ..., n_k \in
      \mathbb{C}\). Zvolíme notaci \(\mathcal{T}^{-1}(\xi_k 
      | x) := \mathcal{D}(\xi_k | x)\). \(\mathcal{D}(n_1,
      n_2, ..., n_k | x) \) je \(\mathcal{D}\) funkce k-tého
      řádu s bazí \(x\).
\end{definition}

V tomto dokumentu se budeme především věnovat 
jednomu typu ksi vektoru \(\xi_k = (2, 3, 4, ..., k-1)\)
. Pro tento typ ksi vektoru využijeme speciální notaci
\(\xi_k = (2, 3, 4, ..., k-1) := \xi_k^*\). Funkci
\(T(\xi_k^* | x) := \eta_k(x)\)
, což budeme nazývat "eta funkce k-tého řádu".

\begin{lemma}[sumační či integrálová reprezentace \(\mathcal{T}\) funkce]
      Pro \(\mathcal{T}( \xi_k | x)\) můžeme odvodit různé identity
      prostřednictvím binomického teorému a jiných již existujících teorémů.
      
      \begin{align}
            \mathcal{T}(\xi_k| x) &= \sum_{v=0}^\infty 
            \binom{\mathcal{T}(\xi_k \setminus \alpha| 
            x)}{v}(\alpha - 1)^v \\
            \mathcal{T}(\xi_k| x) &= \sum_{v=0}^\infty 
            \frac{(\Re(\mathcal{T}(\xi_k \setminus \alpha|
            x)))^v }{v!}\alpha^{i \Im(\mathcal{T}(\xi_k 
            \setminus \alpha| x))}ln^v(\alpha) \\
            \mathcal{T}(\xi_k| x) &= \sum_{v=0}^\infty 
            \frac{(\Im(\mathcal{T}(\xi_k\setminus\alpha|
            x)))^v}{v!}\alpha^{\Re(\mathcal{T}(\xi_k
            \setminus \alpha| x)))}(i \ln(\alpha))^v \\
            \mathcal{T}(\xi_k| x) &= \frac{1}{2\pi i
            \Gamma(-\mathcal{T}(\xi_k \setminus \alpha| 
            x))}\int_{-i \infty + \gamma}^{i \infty + 
            \gamma} \frac{\Gamma(s)\Gamma(-\mathcal{T}(\xi_k
            \setminus \alpha| x) - s)}{(\alpha - 1)^s}ds  
      \end{align}
      
      kde \(0 < \gamma < - \Re(T(\xi_k \setminus 
      \alpha)), |Arg(z)| < \pi\). 
\end{lemma}

Pro \(\mathcal{D}\)-funkci můžeme odvodit
následující analytické pokračování.

\begin{theorem}[Taylorova expanze D-funkce]
      D-funkci může vyjádřit sumou pomocí 
      taylorovy expanze bodě \(a = 2\), pro \(k >1\):

      \begin{align}
            \mathcal{D}(\xi_k | x) = \sum_{n=0}^\infty \frac{(-1)^{n+1}}{2^n 
            \ln(\alpha) n}\mathcal{D}(\xi_k \setminus \alpha | x) - 2 )^n
      \end{align}
\end{theorem}

\begin{proof}
      Definice Taylorovy série činí
      \begin{definition}
            V případě existence všech konečných derivací
            funkce \(f\) v bodě \(a\), potom lze řadu zapsat ve tvaru

            \begin{align}
                  f(x) = \sum_{n=0}^\infty \frac{f^{(n)}(a)}{n!}(x-a)^n
            \end{align}
            
      \end{definition}
            potom pro vytvoření této série pro \(\mathcal{D}\) 
            funkci potřebujeme vzorec pro n-tou derivaci \(\log_k(x)\), tento vzorec činí
      \begin{align}
            \log_k^{(n)}(x) = \frac{(-1)^{1+n}(n-1)!}{x^n \ln(k)}
      \end{align}
      
      potom
      
      \begin{align}
            \log_k(x) &= \sum_{m=0}^\infty
            \frac{(-1)^{1+m}(m-1)!}{2^m 
            \ln(k) m!}(x-2)^m = \\
            & = \sum_{m=0}^\infty (-1)^{1+m}
            \frac{2^{-m}}{m \ln(k)}(x-2)^m 
      \end{align}
      
      aplikujeme-li získanou sérii na \(\mathcal{D}\) funkci
      
      \begin{align}
            \mathcal{D}(\xi_k | x) = \sum_{n=0}^\infty 
            \frac{(-1)^{n+1}}{2^n \ln(\alpha) n}\mathcal{D}(\xi_k
            \setminus \alpha | x) - 2 )^n
      \end{align}
\end{proof}

\begin{definition}[Tau funkce]
      \(\tau\) funkce je pevný bod \(\mathcal{T}\) funkce \(\xi\) 
      vektoru. Tato funkce je symetrická \(\overline{\tau(\xi_k)}=
      \tau(\overline{\xi_k})\). \(\tau\) funkcí rozumíme systém
      \[\mathcal{T}(\xi_k | \tau(\xi_k)) = \mathcal{D}(\xi_k | 
      \tau(\xi_k)) = \tau(\xi_k)\]
      kde \(\xi_k = (n_1, ..., n_k); n_1, ..., n_k \in \mathbb{C}\). 
      \(\tau\) funkce ve tvaru \(\tau(\xi_k)\) je cyklus k-tého řádu.
      Hodnoty tau funkce pro \(\xi_k\) jsou neutrálním prvek pro 
      \(\mathcal{T}\) funkci o vektoru \(\xi_k\). Následně lze
      odvodit následující vztahy pro \(\tau\) funkci

      \begin{align}
            \tau(\xi_k) &= \frac{ln(\mathcal{D}(\xi_{k-1}|\tau(\xi_k
            ))}{ln\|\omega\| + iArg(\omega)} & \\
            \tau(\xi_k) &= \alpha^{\frac{\mathcal{T}(\xi_k \setminus 
            \alpha | \tau(\xi_k))}{2}}= \|\alpha\|^{\frac{\mathcal{T}(
            \xi_k \setminus \alpha | \tau(\xi_k))}{2}} e^{i\mathcal{T}(
            \xi_k \setminus \alpha | \tau(\xi_k))Arg(\alpha)} 
      \label{4}
      \end{align}
\end{definition}

V definici rovnice (\ref{4}) jsou odvozeny od běžných 
rekurzivních vztahů logaritmů a iterativních exponentů.
Z těchto dvou rovnic lze odvodit následující důsledek

\begin{corollary}
      Prostřednictvím rovnic (\ref{4}) lze odvodit
      identitní vztah \(\tau\) funkce k \(\mathcal{D}\) funkci
      
      \begin{equation}\label{R:int}
            \mathcal{D}(\xi_{k-1} | \tau(\xi_k)) =
            \exp\left(\frac{\|\alpha\|^{\frac{\mathcal{T}(\xi_k
                  \setminus \alpha | \tau(\xi_k))}{2}}(ln\|\omega\| +
            i Arg(\omega))}{e^{-i\mathcal{T}(\xi_k \setminus \alpha
            | \tau(\xi_k))Arg(\alpha)} }\right)
      \end{equation}
\end{corollary}
